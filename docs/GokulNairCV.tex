%%%%%%%%%%%%%%%%%%%%%%%%%%%%%%%%%%%%%%%%%
% Medium Length Graduate Curriculum Vitae
% LaTeX Template
% Version 1.1 (9/12/12)
%
% This template has been downloaded from:
% http://www.LaTeXTemplates.com
%
% Original author:
% Rensselaer Polytechnic Institute (http://www.rpi.edu/dept/arc/training/latex/resumes/)
%
% Important note:
% This template requires the res.cls file to be in the same directory as the
% .tex file. The res.cls file provides the resume style used for structuring the
% document.
%
%%%%%%%%%%%%%%%%%%%%%%%%%%%%%%%%%%%%%%%%%

%----------------------------------------------------------------------------------------
%	PACKAGES AND OTHER DOCUMENT CONFIGURATIONS
%----------------------------------------------------------------------------------------

\documentclass[margin]{res} % Use the res.cls style, the font size can be changed to 11pt or 12pt here

\usepackage{helvet} % Default font is the helvetica postscript font
%\usepackage{newcent} % To change the default font to the new century schoolbook postscript font uncomment this line and comment the one above
\usepackage{hyperref}
\setlength{\textwidth}{5.1in} % Text width of the document

\begin{document}

%----------------------------------------------------------------------------------------
%	NAME AND ADDRESS SECTION
%----------------------------------------------------------------------------------------

\moveleft.5\hoffset\centerline{\Large\sc Gokul G. Nair} % Your name at the top
 
\moveleft\hoffset\vbox{\hrule width\resumewidth height 1pt}\smallskip % Horizontal line after name; adjust line thickness by changing the '1pt'
 
\moveleft.5\hoffset\centerline{Center For Applied Mathematics} % Your address
\moveleft.5\hoffset\centerline{Cornell University, NY}
	\moveleft.5\hoffset\centerline{gn234@cornell.edu}

%----------------------------------------------------------------------------------------

\begin{resume}

\section{EDUCATION}  
\textbf{Ph.D. in Applied Mathematics} \hfill 2018 - present\\
Cornell University, Ithaca, NY, United States\\
Advisor: Timothy Healey, Department of Mathematics\\
Expected completion: 2024
\smallskip

\textbf{Masters in Applied Mathematics} \hfill 2018-2021\\
Cornell University, Ithaca, NY, United States\\
Minors: Mathematics, Theoretical Physics
\smallskip

\textbf{Bachelor of Science in Physics} (First Class with Distinction) \hfill 2014 - 2018 \\
Indian Institute of Science, Bangalore, India\\
Thesis Advisors: Chandan Dasgupta \& Sriram Ramaswamy, Department of Physics\\
Thesis Title: \textit{On the Statistics of Extremes in Self-Driven Particles}

\section{VISITING POSITIONS}
\textbf{Visiting Graduate Student} \hfill Feb - April 2023\\
Hausdorff Research Institute for Mathematics,\\
University of Bonn, Bonn, Germany
\smallskip

\textbf{Visiting Scholar} \hfill May - July 2017\\
Department of Engineering Sciences and Applied Mathematics,\\
Northwestern University, Evanston, IL, United States

\section{RESEARCH \\ INTERESTS} 

Calculus of Variations, Nonlinear Elasticity, Applied Analysis, Minimal Surfaces, Complex systems 
 
\section{HONOURS/\\AWARDS}
{\bf Mathematics Instructor Development Certificate}, Cornell (2023)\\
\textit{Awarded to students who complete a series of teaching-related activities.}

{\bf Graduate Student Teaching Award}, Math Dept., Cornell (2022)\\
\textit{Awarded annually to 2-4 students in recognition of the importance of teaching. }

{\bf Cornell Research Travel Grant}, Cornell University (2022-23)\\
\textit{Awarded to Ph.D. students to attend conferences at which they are presenting.}

{\bf Mathematics Teaching Development Fellow}, Cornell University (2022)\\
\textit{Organized teaching seminars and supported TA professional development.}

{\bf First Class with Distinction}, Indian Institute of Science (2018)\\
\textit{Equivalent to ``summa cum laude'' in the Indian honours system.}


{\bf S.N. Bose Fellowship,} Indo-U.S. Science and Technology Forum (2017)\\
\textit{To support Indian undergraduates to pursue research projects in the United States.}

{\bf KVPY Fellowship,} Government of India (2014)\\
\textit{Scholarship encouraging undergraduates to take up research careers in science.}

\section{RESEARCH\\PUBLICATIONS}
\begin{enumerate}
	\item {\sl Energy minimizing configurations for single-director Cosserat shells}, Timothy J. Healey, Gokul G. Nair. Journal of Elasticity (2023) \href{https://arxiv.org/abs/2208.09051}{\texttt{arXiv:2208.09051}}

	\item {\sl Nonlinearly elastic maps: Energy minimizing configurations of membranes on prescribed surfaces}, Timothy J. Healey, Gokul G. Nair. (Under review, SIAM Math. Anal. 2023) \href{https://arxiv.org/abs/2308.02070}{\texttt{arXiv:2308.02070}} 
	
	\item {\sl Stationary curves under the M\"obius-Plateau energy}, Max Lipton, Gokul G. Nair. arXiv preprint (2023) \href{https://arxiv.org/abs/2208.12678}{\texttt{arXiv:2208.12678}} 
	
	\item {\sl Dynamics and synchronization in random networks of coupled phase-oscillators: A graphon approach}, (with Shriya Nagpal, Francesca Parise and Steven Strogatz). (In preparation 2023)
	
	\item {\sl Designing for robustness in electric grids via a general effective resistance measure}, Shriya V. Nagpal, Gokul G. Nair, Francesca Parise, and C. Lindsay Anderson. IEEE TCNS (2022) \href{https://arxiv.org/abs/2201.00929}{\texttt{arXiv:2201.00929}}
	
	\item {\sl Fission-fusion dynamics and group-size-dependent composition in heterogeneous populations}, Gokul G. Nair, Athmanathan Senthilnathan, Srikanth K. Iyer, and Vishwesha Guttal. Physical Review E (2019) \href{https://arxiv.org/abs/1711.06882}{\texttt{arXiv:1711.06882}}
\end{enumerate}

\section{TEACHING EXPERIENCE}

\begin{itemize}
	\item Calculus II, Instructor (Fall 2022)
	\item Partial Differential Equations, Grader (Spring 2022, Fall 2023)
	\item Differential Equations for Engineers, TA (Fall 2021)
	\item Honours Introduction to Analysis I, Grader (Fall 2020)
	\item Multivariable Calculus, TA (Spring 2020)
	\item Differential Equations for Engineers, TA (Fall 2019)
	\item Multivariable calculus for Engineers, TA (Fall 2018 - Spring 2019)
\end{itemize}

\section{TALKS}
\begin{itemize}
	\item \textbf{SIAM Conference on Mathematical Aspects of Materials Science:} \textit{Existence theorems and regularity properties for highly deformable elastic plates and shells} (2024) *\textit{Invited talk}
	\item \textbf{University of Pisa, Department of Civil and Industrial Engineering seminar:} \textit{Existence theorems for highly deformable elastic surfaces} (2023) *\textit{Invited talk}
	\item \textbf{SIAM New York-New Jersey-Pennsylvania Annual Meeting:} \textit{Energy minimizing configurations of highly stretchable elastic surfaces} (2023)
	\item \textbf{Hausdorff Center for Mathematics, Bonn, Workshop on Nonlinear PDEs: Recent Trends in the Analysis of Continuum Mechanics:} \textit{Energy Minimizing Configurations for Single-Director Cosserat Shells} (2023) 
	\item \textbf{Cornell University, Applied Dynamics Seminar:} \textit{A graphon approach to synchronization on large random graphs} (2023)
	\item \textbf{Hausdorff Institute, University of Bonn, Workshop on Variational methods for complex phenomena in solids:} \textit{Energy Minimizing Configurations for Single-Director Cosserat Shells} (2023) *\textit{Invited talk}
	\item \textbf{Hausdorff Institute, University of Bonn, Work group seminar:} \textit{Convex Integration for the $p$-Laplace equation} (2023)
	\item \textbf{Cornell University, Analysis Seminar:} \textit{Energy Minimizing Configurations for Highly Deformable Elastic Surfaces} (2022)
	\item \textbf{University of Ulm, Horizons in Nonlinear PDEs Summer School:} \textit{Energy Minimizing Configurations for Highly Deformable Elastic Surfaces} (2022)
	\item \textbf{Cornell University, Applied Dynamics Seminar:} \textit{Schoen and Yau's proof of the Positive Mass theorem} (2021)
	\item \textbf{Cornell University, Applied Math Student Seminar:} \textit{Introduction to Curvature} (2020)
	\item \textbf{Cornell University, Dynamics Seminar:} \textit{Proving the Uniformization theorem using Ricci flow} (2020)
	\item \textbf{Cornell University, Mathematics Teaching Seminar:} \textit{Promoting Creative Reasoning via Good Questions}, with S.~Ong (2022)
	\item \textbf{Cornell University, Applied Dynamics Seminar:} \textit{On the Dynamics of Power Grids} (2022)
	\item \textbf{Cornell University, REU programme:} \textit{Introduction to Synchronization and the Kuramoto model} (2019)
\end{itemize}

\section{CONFERENCES/\\WORKSHOPS}
\begin{itemize}
	\item SIAM New York-New Jersey-Pennsylvania annual meeting, Newark (2023)
	\item Workshop on Nonlinear PDEs: Recent Trends in the Analysis of Continuum Mechanics, Hausdorff Center for Mathematics, University of Bonn, Germany (2023)
	\item Workshop on Variational Methods for Complex Phenomena in Solids, Hausdorff Institute for Mathematics, University of Bonn, Germany (2023)
	\item Mathematics for Complex Materials Trimester Programme, Hausdorff Institute for Mathematics, University of Bonn, Germany (2023)
	\item Horizons in Nonlinear PDEs Summer School, University of Ulm, Germany (2022)
	\item Communicating Mathematics Conference, Cornell University (2022)
	\item STEM Communication Workshop, Alan Alda Center for Communicating Science (2021)
\end{itemize}

\section{SERVICE}
\begin{itemize}
	\item Facilitator, Mathematics TA training programme (2023)
	\item President, Cornell SIAM chapter (2021-2022)
	\item Organizer, Mathematics teaching seminar (Fall 2022)
	\item Organizer, Applied Mathematics Student Seminar (2020-2022)
	\item Mentor, Directed Reading Programme, Cornell Department of Mathematics (2020-2022)
	\item Expanding Your Horizons Volunteer, Cornell University (2022)
	\item Facilitator, Mathematics TA training programme (2020)
\end{itemize}

\section{HUMAN \\ LANGUAGES}
\begin{itemize}
	\item Native proficiency: English, Malayalam
	\item Fluent: Hindi, Kannada
	\item Limited proficiency: Tamil, Sanskrit
\end{itemize}

\section{COMPUTER \\ LANGUAGES}
C, C++, Python (Numpy, Scipy, Fenics), Mathematica, \LaTeX, HTML, CSS

\end{resume}
\end{document}